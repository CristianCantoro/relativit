\chapter[Costanti]{Costanti\protect\footnote{Tratto da \cite{mohr:1187}}}
\section{Costanti matematiche}
\begin{small}
\begin {tabular}{ll}
\hline
Simbolo&Valore\\ \hline
$\pi$&$3.141592653591403978482542414219279663919893234825\ldots$\\
$\pi^2$&$9.869604401089358618834490999876151135313699407240\ldots$\\
$\sqrt{\pi}$&$1.772453850905516027298167483341145182797549456122\ldots$\\
$e$&$2.718281828458563411277850606202642376785584483618\ldots$\\
$e^2$&$7.389056098930650227230427460575007813180315570551\ldots$\\
$\sqrt{e}$&$1.648721270700128146848650787814163571653776100710\ldots$\\
$\pi/e$&$1.155727349791723336742691256256860982838416464162\ldots$\\
$e/\pi$&$0.865255979431664940921802827901482365570974166378\ldots$\\
$4\pi\times 10^{-7}$&$1.256637061435917295385057353311801153678867\ldots\times 10^{-6}$\\
\hline
\end{tabular}
\section{Costanti universali}

\begin{tabellacostanti}
%aggiornata al 2006
velocità della luce nel vuoto &$c$, $c_0$
&$299792458$&\per\meter\second&esatto\\
Permeabilità magnetica del vuoto&$\mu_0$&$4\pi\times
10^{-7}=12.566370614\ldots\times 10^{-7}$&\newton\Square\ampere&esatto\\
Costante dielettrica del vuoto
$1/\mu_0c^2$&$\varepsilon_0$&$8.854187817\ldots\times
10^{-12}$&\farad\per\metre&esatto\\
Impedenza del vuoto $\sqrt{\mu_0/\varepsilon_0}=\mu_0 c$&$Z_0$&$376.730313461\ldots $&\ohm&esatto\\
Costante di gravitazione universale&$G$&$6.67248(67)\times
10^{-11}$&\cubic\metre\per\kilogram\Square\second&$1.0\times 10^{-4}$\\
&$G/\hbar c$&$6.70881(67)\times
10^{-39}$&$(\giga\electronvolt\per c^2)^{-2}$&$1.0\times 10^{-4}$\\
Costante di Planck&$h$&$6.62606896(33)\times
10^{-34}$&\joule\second&$5.0\times 10^{-8}$\\
&&$4.13566733(10)\times 10^{15}$&\electronvolt\second&$2.5\times
10^{-8}$\\
$h/2\pi$&$\hbar$&$1.054571628(53)\times
10^{-34}$&\joule\second&$5.0\times 10^{-8}$\\
&&$6.58211899(16)\times10^{-16}$&\electronvolt\second&$2.5\times
10^{-8}$\\
&&$197.3266931(49)$&\mega\electronvolt\femto\meter&$2.5\times 10^{-8}$\\
Massa di Planck $(\hbar c/G)^{1/2}$&$m_P$&$2.17644(11)\times
10^{-8}$&\kilogram&$5.0\times 10^{-5}$\\
&$m_P c^2$&$1.220892(61)\times 10^{19}$&\giga\electronvolt&$5.0\times 10^{-5}$\\
Temperatura di Plank $(\hbar c^5/G)^{1/2} /k$&$T_P$&$1.416785(71)\times 10^{32}$&\kelvin&$5.0\times 10^{-5}$\\
Lunghezza di Planck $\hbar/m_pc=(\hbar G/c^3)^{1/2}$&$l_P$&$1.616252(81)\times 10^{-35}$&\meter&$5.0\times 10^{-5}$\\
Tempo di Planck $l_P/c=(\hbar G/c^5)^{1/2}$&$t_P$&$5.39124(27)\times 10^{-44}$&\second&$5.0\times 10^{-5}$\\
\end{tabellacostanti}

%tutte da aggiornare
\section{Costanti elettromagnetiche}
\begin{tabellacostanti}
Carica elementare&$e$&$1.602176487(40)\times 10^{-19}$&\coulomb&$2.5\times 10^{-8}$\\
 &$e/h$&$2.417989454(60)\times 10^{14}$&\ampere\per\joule&$2.5\times 10^{-8}$\\
Flusso ma\-gne\-ti\-co quantico
$h/2e$&$\Phi_0$&$2.06783372(18)\times
10^{-15}$&\weber& $8.5\times 10^{-8}$\\
Conduttanza quantica $2e^2/h$&$G_0$&$7.748091696(28)\times
10^{-5}$&\second&$3.7\times 10^{-9}$\\
Inverso della conduttanza
quantica&$G_0^{-1}$&$12906.403786(47)$&\ohm&$3.7\times
10^{-9}$\\
Costante di Josephson $2e/h$&$K_J$&$483597.898(19)\times
10^9$&\hertz\per\volt&$3.9\times 10^{-8}$\\
Costante di von Klitzing $h/e^2=\mu_0
c/2\alpha$&$R_K$&$25814.807572(95)$&\ohm&$3.7\times
10^{-9}$\\
Magnetone di Bohr $e\hbar/2m_e$&$\mu_B$&$927.400899(37)\times 10^{-26}$&\joule\per\tesla&$4.0\times 10^{-8}$\\
                                      &&$5.788381749(43)\times
                                      10^{-5}$&\electronvolt\per\tesla&$7.3\times
                                      10^{-9}$\\
&$\mu_B/h$&$13.99624624(56)\times
10^9$&\hertz\per\tesla&$4.0\times 10^{-8}$\\
&$\mu_B/hc$&$46.6864524(19)$&\per\metre\per\tesla&$4.0\times
10^{-8}$\\
&$\mu_b/k$&$0.4717131(12)$&\kelvin\per\tesla&$1.7\times
10^{-6}$\\
\end{tabellacostanti}

\section{Costanti atomiche e nucleari}
\subsection{Generali}

\begin{tabellacostanti}
Costante di struttura fine $e^2/4\pi\varepsilon_0\hbar
c$&$\alpha$&$7.297352533(27)\times 10^{-3}$& &$3.7\times 10^{-9}$\\
Inverso della costante di struttura fine
&$\alpha^{-1}$&$137.03599976(50)$& &$3.7\times 10^{-9}$\\
Costante di Rydberg
$\alpha^2m_ec/2h$&$R_\infty$&$10973731.568549(83)$&\per\metre&$7.6\times
10^{-12}$\\
&$R_{\infty} c$&$3.289841960368(25)\times
10^{15}$&\hertz&$7.6\times 10^{-12}$\\
&$R_{\infty} hc$&$2.17987190(17)\times
10^{-18}$&\joule&$7.8\times 10^{-8}$\\
&$R_{\infty} hc$&$13.60569172(53)$&$\electronvolt$&$3.9\times 10^{-8}$\\
\end{tabellacostanti}

\subsection{Elettrone, $\textrm{e}^-$}
\begin{tabellacostanti}
massa elettrone& $m_{\mathrm{e}}$&$9.10938188(72)\times 10^{-31}$&\kilogram&$7.9\times 10^{-8}$\\
&&$5.485799110(12)\times 10^{-4}$&\atomicmass&$2.1\times
10^{-9}$\\
energia equivalente&$m_{\mathrm e}c^2$&$8.18710414(64)\times
10^{-14}$&$\joule$&$7.9\times 10^{-8}$\\
&&$0.510998902(21)$&\mega\electronvolt&$4.0\times 10^{-8}$\\
rapporto massa elettrone muone&$m_\mathrm
e/m_\mu$&$4.83633210(15)\times 10^{-3}$&&$3.0\times
10^{-8}$\\
rapporto massa elettrone tau&$m_\mathrm
e/m_\tau$&$2.87555(47)\times
10^{-4}$&&$1.6\times 10^{-4}$\\

rapporto massa elettrone protone&$m_\mathrm
e/m_\mathrm{p}$&$5.446170232(12)\times 10^{-4}$&&$2.1\times
10^{-9}$\\

rapporto massa elettrone neutrone&$m_\mathrm e/m_\mathrm
n$&$5.438673462(12)\times 10^{-4}$&&$2.2\times
10^{-9}$\\

rapporto massa elettrone deuterone&$m_\mathrm e/m_\mathrm d$&$2.7244371170(58)\times 10^{-4}$&&$2.1\times 10^{-9}$\\

rapporto massa elettrone particella alfa&$m_\mathrm e/m_\mathrm \alpha$&$1.3709335611(29) \times 10^{-4}$&&$2.1\times 10^{-9}$\\

rapporto carica massa&$\mathrm{e}/m_\mathrm{e}$&$-1.758820174(71)\times  10^{11}$&\coulomb\per\kilogram&$4.0\times 10^{-8}$\\

massa molare $N_\mathrm{A}m_\mathrm{e}$&$M(\mathrm{e}),M_\mathrm{e}$&$5.485799110(12)\times 10^{-7}$&\kilogram\per\mole&$2.1\times 10^{-9}$\\

lunghezza d'onda di Compton $h/m_\mathrm{e}c$&$\lambda_\mathrm{C}$&$2.426310215(18)\times 10^{-12}$&\meter&$7.3\times 10^{-9}$\\
$\:\:\lambda_\mathrm{C}/2\pi=\alpha a_0=\alpha^2/4\pi R_\infty$&&$386.1592642(28)\times 10^{-15}$&\meter&$7.3\times 10^{-9}$\\

raggio classico dell'elettrone&$r_\mathrm{e}$&$2.817940285(31)\times 10^{-15}$&\meter&$1.1\times 10^{-8}$\\

\end{tabellacostanti}

\subsection{Muone, $\mu^{-}$}
\begin{tabellacostanti}
massa&$m_\mu$&$1.88353109(16)\times
10^{-28}$&\kilogram&$8.4\times 10^{-8}$\\
&&$0.1134289168(34)$&\atomicmass&$3.0\times 10^{-8}$\\
energia equivalente&$m_\mu c^2$&$1.69283332(14)\times
10^{-11}$&$\atomicmass$&$8.4\times 10^{-8}$\\
&&$105.6583568(52)$&$\mega\electronvolt$&$4.9\times 10^{-8}$\\
\end{tabellacostanti}

\subsection{Tau, $\tau^-$}
\begin{tabellacostanti}
massa&$m_\tau$&$3.16788(52)\times 10^{-27}$&\kilogram&$1.6\times 10^{-4}$\\
\end{tabellacostanti}



\subsection{Protone, p}
\begin{tabellacostanti}
massa&$m_\mathrm p$&$1.67262158(13)\times
10^{-27}$&\kilogram&$7.9\times 10^{-8}$\\
rapporto massa protone--elettrone&$m_{\mathrm{p}}/m_{\mathrm{e}}$&$1836.1526675(39)$&&$2.1\times 10^{-9}$\\
rapporto massa protone--neutrone&$m_\mathrm{p}/\mathrm{n}$&$0.99862347855(58)$&&$5.8\times 10^{-10}$\\
rapporto carica--massa&$e/m_p$&$9.57883408(38)\times 10^7$&$\coulomb\per\kilogram$&$4.0\times 10^{-8}$\\
\end{tabellacostanti}

\subsection{Neutrone, n}
\begin{tabellacostanti}
massa&$m_\mathrm n$&$1.67497216(13)\times
10^{-27}$&\kilogram&$7.9\times 10^{-8}$\\
\end{tabellacostanti}

\subsection{Particella alfa, $\alpha$}
\begin{tabellacostanti}
massa&$m_\alpha$&$6.64465598(52)\times 10^{-27}$&\kilogram&$7.9\times 10^{-8}$\\
rapporto massa particella alpha--elettrone&$m_\alpha/m_\mathrm{e}$&$7294.299508(16)$&&$2.1\times 10^{-9}$\\
\end{tabellacostanti}

\section{Costanti fisico--chimiche}



\begin{tabellacostanti}
Numero di Avogadro&$N_A, L$&$6.02214199(47)\times
10^{23}$&\per\mole&$7.9\times 10^{-8}$\\
Costante di massa atomica
$m_\atomicmassunit=\frac{1}{12}m_{{}^{12}_{}{\textrm{C}}}=1\atomicmass=\si{1E-3}{\kilogram\per\mole\per N_A}$&$m_\atomicmassunit$&$1.66053873(13)\times{10^{-27}}$&\kilogram&$7.9\times
10^{-8}$\\
energia equivalente dell'unità di massa
atomica&$m_{\atomicmass}c^2$&$1.49241778(12)\times
10^{-10}$&$\joule$&$7.9\times 10^{-8}$\\
&&$931.494013(37)$&$\mega\electronvolt$&$4.0\times 10^{-8}$\\
Costante di Faraday
$N_Ae$&$F$&$964853415(39)$&\coulomb\per\mole&$4.0\times
10^{-8}$\\
Mole di Plank&$N_Ah$&$3.990312689(30)\times
10^{-10}$&\joule\second\per\mole&$7.6\times
10^{-9}$\\
&$N_Ahc$&$0.11962656492(91)$&\joule\metre\per\mole&$7.6\times
10^{-9}$\\

Costante dei gas&$R$&$8.314472(15)$&\joule\per\mole\per\kelvin&$1.7\times
10^{-6}$\\

Costante di Boltzmann $R/N_A$&$k$&$1.3806503(24)\times
10^{-23}$&\joule\per\kelvin&$1.7\times
10^{-6}$\\

&&$8.617342(15)\times 10^{-5}$&\electronvolt\per\kelvin&$1.7\times
10^{-6}$\\




\end{tabellacostanti}
\end{small}
