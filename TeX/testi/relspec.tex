\chapter{Relatività speciale\index{relatività!speciale|(}\index{relatività!ristretta|see{speciale}}}
\minitoc

La cinematica sviluppata da Galileo e la meccanica sviluppata da Newton, che costituiscono le fondamenta della cosiddetta 
fisica classica, hanno riscosso enormi successi: ricordiamo in particolare la teoria copernicana del moto dei pianeti 
nel sistema solare e il ricorso alla cinematica per spiegare certe proprietà dei gas scoperte sperimentalmente, la 
teoria cinetica dei gas.
Rimangono tuttavia alcune categorie di fenomeni scoperti sperimentalmente che non possono essere interpretati alla luce 
di queste teorie classiche.

Nel 1905 Einstein espose la sua teoria della relatività ristretta partendo da un esperimento ideale da lui immaginato. 
A 16 anni, studiando la teoria dell' elettromagnetismo, aveva escogitato un paradosso: se vi muoveste alla velocità della luce 
parallelamente a un raggio di luce che viaggia nello spazio vuoto, osservereste un campo ``statico'' di elettricità e dei 
tracciati di campi magnetici. Ma Einstein ben sapeva che l'esistenza simultanea di campi elettrostatici e di campi magnetici 
violerebbe la teoria dell'elettromagnetismo. 
Bisognava far convivere le equazioni di Maxwell\index{Maxwell} con quelle di Newton.

Einstein si trovò di fronte a un' alternativa per risolvere questo paradosso: era sbagliata la teoria elettromagnetica
oppure lo era la cinematica classica, che ammette l'esistenza di un osservatore che può viaggiare solidale con un raggio luminoso? 

Einstein puntò sulla teoria elettromagnetica e andò in cerca di una variante alle teorie cinematiche di Newton e Galileo. 
Questa nuova cinematica, che sta alla base della relatività ristretta, vieta a qualsiasi osservatore (dotato di massa) 
di seguire solidarmente un raggio di luce.

Infine, vale la pena ricordare che quando i risultati sperimentali della fisica classica e della teoria della 
relatività appaiono in disaccordo, ha sempre vinto la relatività.

\section{Postulati di Einstein\index{Einstein!postulati di}\index{postulato!di Einstein}}
La teoria della relatività si fonda su due postulati che riportiamo qui di seguito:
\begin{post}[Principio di relatività speciale]
Tutte le leggi fisiche sono invarianti per traslazioni uniformi
\end{post}
\begin{post}[Invarianza della velocità della luce]
La velocità della luce nel vuoto è uguale in tutti i sistemi di riferimento
\end{post}

Dobbiamo trovare una la forma di una generica trasformazione che rispetti questi due postulati. 
Lo facciamo nella prossima sezione.

\subsection{Trasformate di Lorentz\index{trasformate!di Lorentz|(}\index{trasformazioni!di Lorentz|(}\index{Lorentz}}
Le trasformate i Lorentz sostituiscono le trasformate di Galileo della meccanica classica.
Un'altra caratteristica desiderabile per queste trasformazioni è che per velocità basse rispetto a quelle della luce 
le trasformate di Lorentz tendono a quelle di Galileo, infatti ci aspettiamo che la meccanica classica, che descrive così
bene il moto di treni, automobili, palle di cannone e altri oggetti ordinari, sia contenuta in questa teoria più estesa.
Ricaviamo nel seguito la forma esplicita di queste trasformazioni.

La funzione che trasforma le coordinate di un sistema di riferimento 
a quelle di un altro deve essere lineare. Un'applicazione si dice lineare se gode di due proprietà:
\begin{itemize}
 \item omogeneità: $f(\alpha \ve x) = \alpha f(\ve x)$, $\alpha \in \mathchar{R}$ 
 \item additività: $f(\ve x + \ve y) = f(\ve x) + f(\ve y)$
\end{itemize}
 
Infatti, immaginando che l'applicazione non sia lineare, per esempio: $x'=kx^2$, otteniamo dei risultati assurdi. 
Se mettessimo una sbarra tra $0$ e $1$ questa misura $\Delta x=1-0=1$, 
nel nuovo sistema di riferimento misura: $k\cdot 1^2-k\cdot 0^2=k$.
Se ora mettessimo la sbarra tra $2$ e $1$ nel primo sistema di riferimento misurerà ancora $1$, mentre nel secondo ù
si avrebbe $k(2^2-1^2)=3k$ ovvero spostando la sbarra nel secondo sistema di riferimento la sua lunghezza cambia: 
assurdo per l'ipotesi di omogeneità dello spazio; inoltre l'applicazione deve essere anche additiva, 
perché la lunghezza di due sbarre affiancate deve essere la somma dell'unica sbarra avente lunghezza 
uguale nel primo sistema di riferimento alla somma delle lunghezze delle due sbarre nel primo sistema di riferimento. 

Consideriamo il sistema di riferimento $Oxy$ fermo e $O'x'y'$ che si muove rispetto ad $O$ di moto rettilineo 
uniforme con velocità di trascinamento $u$ nella direzione dell'asse $x$ crescente. L'applicazione è lineare, 
quindi (in due dimensioni spaziali), in generale:
\[\left\{
\begin{array}{l}
x'=\lambda_{11}x+\lambda_{12}y+\lambda_{14}t\\
y'=\lambda_{21}x+\lambda_{22}y+\lambda_{24}t\\
t'=\lambda_{41}x+\lambda_{42}y+\lambda_{44}t\\\end{array}\right.
\]
Dove troviamo un sistema con 9 incognite. Facciamo alcune ipotesi aggiuntive:
\begin{itemize}
\item[-] al tempo $t=t'=0$ gli orologi coincidono e anche i sistemi di riferimento $O\equiv O'$, in particolare per $t=0$ e $x=0$ segue $x'=0$ cioè
\[x'=\lambda_{11}x+\lambda_{12}y+\lambda_{14}t\]
\[0=\lambda_{11}\cdot 0+\lambda_{12}y+\lambda_{14}\cdot 0\]
\[\lambda_{12}=0\]
\item[-] come sopra:
\[t'=\lambda_{41}x+\lambda_{42}y+\lambda_{44}t\]
\[0=0+\lambda_{42}y+0\]
\[\lambda_{42}=0\]
\item[-] la traslazione del sistema in moto è orizzontale, quindi $y'=0\Leftrightarrow y=0$ sempre.
\[\lambda_{21}=0\]
\[\lambda_{24}=0\]

\end{itemize}

Se mettiamo una sbarra verticalmente con un estremo in $O'$, questo misura che la lunghezza è $1$. 
La sbarra si muove con $O'$ e quindi questa è la lunghezza a riposo. Questa lunghezza a riposo dovrà essere uguale anche se è osservata da $O$. Mettiamo la sbarra ferma rispetto ad $O'$, quindi $O$ vede una lunghezza:
\[y=\frac{1}{\lambda_{22}}\]
mettiamo ora la sbarra ferma rispetto a $O$, $O'$ vede una lunghezza:
\[y'=\lambda_{22}\]
Le due situazioni sono simmetriche e quindi non esistendo un sistema privilegiato le due grandezze devono essere uguali:
\[\lambda_{22}=\frac{1}{\lambda_{22}}\qquad\lambda^2_{22}=1\]
Si scarta la soluzione negativa, in quanto si vuole che gli assi abbiano lo stesso verso.

\parbox[]{\textwidth}{
Seguiamo il moto dell'origine del sistema $O'$. Qui, per ogni istante:
\[x'=0\]
Per $O$ lo stesso moto è descritto da:
\[x=ut\qquad x-ut=0\]
Ci si aspetta allora un'equazione del tipo
\[x'=\lambda(x-ut)\]
}

A $t=0$, $O\equiv O'$, $t=t'$ scatta un'onda luminosa dalle origini, questa ha un fronte sferico con raggio $R$ e per il secondo principio di Einstein ha velocità $c$ uguale in tutti e due i sistemi. Il sistema $O'$ sta arrivando da sinistra, passa in coincidenza con $O$, si sincronizzano gli orologi e parte il fronte d'onda. I due osservatori osservano un fronte d'onda che ha centro nei propri sistemi di riferimento, anche se non sono più coincidenti:
\[x^2+y^2+z^2=R^2=(ct)^2\]
\[x'^2+y'^2+z'^2=R'^2=(ct')^2\]
\[x^2+y^2+z^2-ct^2=0=x'^2+y'^2+z'^2-(ct')^2\]
\[x^2-c^2t^2=x'^2-c^2t'^2=\lambda^2(x-ut)^2-c^2(\lambda_{41}x+\lambda_{44}t)^2=\]
\[=\lambda^2x^2+\lambda^2 u^2t^2-2\lambda^2 xut-c^2\lambda_{41}^2x^2-c^2\lambda_{44}^2t^2-2c^2\lambda_{41} x\lambda_{44} t\]
Risolvendo in $\lambda^2$, $\lambda_{41}^2$ e $\lambda_{44}^2$ si ottiene un sistema di tre incognite e tre equazioni:
\[\left\{
\begin{array}{l}
1=\lambda^2-c^2\lambda_{41}^2\\
-c^2=\lambda^2u^2-c^2\lambda_{44}^2\\
0=-2\lambda^2u-2c^2\lambda_{41}\lambda_{44}\\
\end{array}
\right.\quad \left\{
\begin{array}{l}
\lambda_{41}^2=\frac{\lambda^2-1}{c^2}\\
\lambda_{44}^2=\frac{c^2+\lambda^2u^2}{c^2}\\
-\lambda^2u=c^2\lambda_{41}\lambda_{44}\\

\end{array}\right.\]
Ci ricordiamo dell'ultima espressione che $\lambda_{41}\lambda_{44}<0$
\[\lambda^4u^2=c^4\lambda_{41}^2\lambda_{44}^2=c^4\frac{\lambda^2-1}{c^2}\frac{c^2+\lambda^2u^2}{c^2}=\]
\[=(\lambda^2-1)(c^2+\lambda^2u^2)=-c^2-\lambda^2u+\lambda^2c^2+\lambda^4u^2\]
\[0=-c^2-\lambda^2u^2+\lambda^2c^2\]
\[\lambda^2=\frac{c^2}{c^2-u^2}=\frac{1}{1+\frac{u^2}{c^2}}\]
\[\lambda=\dfrac{1}{\sqrt{1-\left(\dfrac{u}{c}\right)^2}}=\dfrac{1}{\sqrt{1-\beta^2}}=\gamma\]
\[\lambda_{41}^2=\frac{\lambda^2-1}{c^2}=\frac{\beta^2}{c^2(1-\beta^2)}\]
\[\lambda_{41}=\pm\frac{\beta}{c\sqrt{1-\beta^2}}\]
\[\lambda_{44}^2=\frac{c^2+\lambda^2u^2}{c^2}=\frac{1}{1-\beta^2}\]
\[\lambda_{44}=\pm\frac{1}{\sqrt{1-\beta^2}}\]
Diamo segno positivo a $\lambda_{44}$ e negativo a $\lambda_{41}$, in quanto se consideriamo un istante molto prossimo al momento della partenza allora $x\rightarrow 0$, $t'$ deve essere positivo (il tempo scorre nello stesso verso) e quindi $\lambda_{44}>0$.

Nei calcoli precedenti abbiamo definito le quantità:
\begin{equation}
\beta = \dfrac{v}{c} 
\end{equation}

\begin{equation}
 \gamma = \dfrac{1}{\sqrt{1-\beta^2}} = \dfrac{1}{\sqrt{1 - \frac{v^2}{c^2}}}
\end{equation}

Quindi alla fine le trasformazioni cercate sono le seguenti:
\begin{equation}
\left\{
\begin{array}{l}
x^\prime=\dfrac{x-vt}{\sqrt{1-\beta^2}}\\
y^\prime=y\\
z^\prime=z\\
t^\prime=\dfrac{t-\frac{\beta}{c}x}{\sqrt{1-\beta^2}}
\end{array}\right.
\end{equation}

\subsection{Trasformate di Lorentz e Galileo}
Verifichiamo che le trasformazioni trovate rispettano quel ``principio di corrispondenza'' accennato sopra, ovvero che la meccanica
classica e le trasformazioni di Galileo vengano riprodotte per velocità ``ordinarie'', questo equivale a dire che se $u\ll c$,
cioè se $\beta\rightarrow 0$ allora le trasformazioni di Lorentz si riducono a quelle di Galileo.
\[\left\{
\begin{array}{l}
x^\prime=\dfrac{x-vt}{\sqrt{1-\beta^2}}\\
y^\prime=y\\
z^\prime=z\\
t^\prime=\dfrac{t-\frac{\beta}{c}x}{\sqrt{1-\beta^2}}
\end{array}\right.
\quad \stackrel{\beta\rightarrow 0}{\longrightarrow} \quad
\left\{
\begin{array}{l}
x^\prime=x-ut\\
y'=y\\
z'=z\\
t'=t\\
\end{array}\right.\]
\index{trasformate!di Lorentz|)} \index{trasformazioni!di Lorentz|)}

\subsection{Le trasformazioni di Lorentz}
Lorentz ha quindi cercato una trasformazione di coordinate adatta per mantenere l’invarianza in forma delle equazioni di Maxwell. 
Riassumiamo le trasformazioni (riportando una sola dimensione spaziale):
\begin{equation}\label{Lorentz_riass}
  \left\{\begin{array}{ll}
   x' = \gamma \left(x - vt \right) \\
   t' = \gamma \left(t - \dfrac{v}{c^2}x \right)
  \end{array}\right.
\end{equation}

Come notiamo, Lorentz ha apportato due modifiche alle trasformazioni
di Fitzgerald. Ha lasciato invariata la contrazione delle lunghezze, ma ha
aggiunto una dilatazione dei tempi. Il fattore davanti a $t$ (inserito per
analogia) non bastava infatti per avere l’invarianza in forma: Lorentz ha
dovuto anche introdurre il termine $- \frac{v}{c^2} x$ (che è a tutti gli effetti un tempo)
nell'equazione dei tempi.

Con questa nuova trasformazione si verifica che:
\begin{equation}
\left\{
  \begin{array}{ll}
  \dfrac{\partial E}{\partial t} = - \gamma v \dfrac{\partial E'}{\partial x'} - \gamma \dfrac{\partial E'}{\partial t'} \\
  \dfrac{\partial E}{\partial x} = \gamma \dfrac{\partial E'}{\partial x'} - \dfrac{\gamma v}{c^2}\dfrac{\partial E'}{\partial t'}                
  \end{array}\right.
\end{equation}

e derivando nuovamente:
\begin{equation}
\left\{
  \begin{array}{ll}
  \dfrac{\partial^2 E}{\partial t^2} = - \gamma^2 v^2 \dfrac{\partial^2 E'}{\partial x'^2} - 2 \gamma v \dfrac{\partial^2 E'}{\partial x' \partial t'} + \gamma^2 \dfrac{\partial^2 E'}{\partial t'^2}\\
  \dfrac{\partial^2 E}{\partial x^2} = \gamma^2 \dfrac{\partial^2 E'}{\partial x'^2} - 2 \dfrac{\gamma^2 v}{c^2}\dfrac{\partial E'}{\partial x' \partial t'} + \dfrac{\gamma^2 v^2}{c^4}\dfrac{\partial E'}{\partial x' \partial t'}
  \end{array}\right.
\end{equation}
Le derivate miste si semplificano. Inoltre sapendo che sono verificate le seguenti identità:
\begin{equation}
\left\{
  \begin{array}{ll}
  \dfrac{v^2}{c^2 \left(1 - \frac{v^2}{c^2}\right)} - \dfrac{1}{\left(1 - \frac{v^2}{c^2}\right)} = -1 \\
  \dfrac{1}{c^2 \left(1 - \frac{v^2}{c^2}\right)} - \dfrac{v^2}{c^4 \left(1 - \frac{v^2}{c^2}\right)} = \dfrac{1}{c^2} \\
  \end{array}\right.
\end{equation}
otteniamo l’equazione delle onde nelle nuove coordinate:
\begin{equation}
 \frac{1}{c^2} \dfrac{\partial^2 E}{\partial t^2} - \dfrac{\partial^2 E}{\partial x^2} = 0
\end{equation}

equazione che ha conservato intatta la forma che aveva come conseguenza
delle equazioni di Maxwell. Le trasformazioni di Lorentz nascono proprio
per mantenere questa invarianza in forma.

\section{Effetti relativistici\index{effetti relativistici}}
In questa sezioni presenteremo alcuni particolari effetti relativistici che si allontanano dal ``senso comune'' e sono peculiari
della teoria della relatività, essi sono:
\begin{itemize}
 \item la contrazione delle lunghezze;
 \item la dilatazione dei tempi;
 \item la perdita del concetto di simultaneità assoluta;
 \item la composizione delle velocità;
\end{itemize}

\subsection{Contrazione delle lunghezze\index{contrazione delle lunghezze}}
\begin{figure}[htbp]
   \centering
   \includegraphics[scale=0.5]{immagini/relspec/beta_gamma}
   \caption{Variazione di $\gamma$ in funzione di $\beta$.}
\end{figure}
Consideriamo l'effetto delle trasformazioni di Lorentz sulla misura di lunghezze.
Posizioniamo un'asta con estremi $A$ e $B$ ferma rispetto ad $O'$, 
sistema di riferimento inerziale con velocità $u$ rispetto ad $O$.
$O$ misura nello stesso istante le posizione di $A$ e di $B$, quindi $t_A\equiv t_B$:
\[\left\{
\begin{array}{l}
x'_A=\frac{x_A-ut_A}{\sqrt{1-\beta^2}}\\
x'_B=\frac{x_B-ut_B}{\sqrt{1-\beta^2}}\\
\end{array}
\right.\]
\begin{align*}
l' &= x'_B-x'_A=\frac{x_B-x_A-ut_B+ut_A}{\sqrt{1-\beta^2}}= \\
&= \frac{x_B-x_A-u(t_B-t_A)}{\sqrt{1-\beta^2}}=\frac{x_B-x_A}{\sqrt{1-\beta^2}}\\
&=\gamma l
\end{align*}
Quindi abbiamo:
\begin{equation}
l=\frac{l'}{\gamma}
\end{equation}
Considerando che $0<\beta<1$ il che implica $\gamma>1$ si ha che $l<l'$, abbiamo che $O$ osserva un oggetto (in moto) 
la cui lunghezza è contratta rispetto alla lunghezza a riposo.

\subsection{Dilatazione dei tempi\index{dilatazione dei tempi}}
Consideriamo ora eventi osservati da $O'$ nello stesso spazio ($x'_A\equiv x'_B$) a distanza di tempo $\tau'=t'_B-t'_A$.
\[t'=\gamma\left(t-\frac{\beta}{c}x\right)\]
Per simmetria posso dire direttamente che:
\[t=\gamma\left(t'+\frac{\beta}{c}x'\right)\]
quindi abbiamo che:
\[t_A=\gamma\left(t'_A+\frac{\beta}{c}x'_A\right)\]
\[t_B=\gamma\left(t'_B+\frac{\beta}{c}x'_B\right)\]
e l'intervallo temporale tra i due eventi $\tau$ è dato da:
\begin{equation}
\tau=t_B-t_A=\gamma\left(t'_B-t'_A+\frac{\beta}{c}\left(x'_B-x'_A\right)\right)=\gamma\left(t'_B-t'_A\right)=\gamma\tau'
\end{equation}
Essendo $\gamma>1$ si ha che $\tau>\tau'$ cioè secondo $O$ il tempo misurato da $O'$ scorre più lentamente.

\begin{Es}[produzione e decadimento di mesoni $\pi^+$]
Se lanciamo un protone a colpire un altro protone fermo si può verificare la seguente ``reazione'':
\[p+p\rightarrow p+n+\pi^+\]
Durante la quale viene prodotto un mesone $\pi$ o pione. Questo decade nel seguente modo:
\[\pi^+ \rightarrow \mu ^+ + \nu_{\mu}\]

Se ci poniamo nel sistema di riferimento del mesone (ossia se lo cavalchiamo)
osserviamo un tempo di vita media $\tau'=\SI{1.88E-8}{\second}$. Il mesone ha rispetto al laboratorio $\beta=0.99$. Il laboratorio dalla distanza percorsa dal mesone calcola una vita media diversa:
\[\tau=\gamma\tau'=\frac{\tau'}{\sqrt{1-\beta^2}}\simeq 7.1\tau'=\SI{1.3E-7}{\second} \]
La distanza percorsa dal mesone, prima di decadere, osservata dal laboratorio è quindi:
\[d=\tau u=\tau \beta c\simeq \SI{39}{\metre} \]
\end{Es}

\subsection{Perdita del concetto di simultaneità assoluta\index{simultaneità!}}

\begin{figure}[htbp]
   \centering
   \includegraphics[scale=0.2]{immagini/relspec/pistoleri}
   \caption{Due pistoleri relativistici.}
   \label{pistoleri}
\end{figure}

I due pistoleri (Fig.\@ \ref{pistoleri}) sparano quando la luce gli raggiunge. 
Essi sono in un sistema solidale con la luce e quindi vedono contemporaneamente la luce della lampadina. 
Per il capostazione ``fermo'' invece è il pistolero di sinistra ad essere investito per prima dalla luce, 
quindi il pistolero a destra muore, mentre quello di sinistra no.
Gli eventi simultanei per $O'$ non sono tali per $O$, bisogna rivedere il concetto di tempo assoluto.

\subsection{Composizione delle velocità\index{composzione delle velocità!relativistiche}}
Come si potrà immaginare la composizione delle velocità non segue più le trasformazioni di Galileo. 
Infatti, supponiamo di osservare una astronave che si muove con $v = 0.8 c$ rispetto ad un sistema di 
riferimento $O$ e contestualmente abbiamo un sistema di riferimento
$O'$ che si muove con velocità $u = 0.8 c$ in direzione opposta a quella dell'astronave.
Le trasformazioni di Galileo ci direbbero che l'astronave $O'$ misura per $O$ la velocità:
\[
 v' = v - u
\]
ossia $v' = 0.8c + 0.8c = 1.6c$ una velocità maggiore di quella della luce, ma questo è impossibile! È necessario
cambiare la legge di composizione delle velocità.

La velocità rimane sempre la derivata del vettore posizione: $v_x=\frac{\ud x}{\ud t}$, 
quindi andiamo a calcolare queste derivate nel sistema traslato per trovare le leggi di composizione delle velocità:
\begin{equation}\label{composizione_vel_x}
\begin{split}
  v'_{x'} &= \frac{\ud x'}{\ud t'}=\frac{\ud x'}{\ud t}\frac{\ud t}{\ud t'}=\frac{\ud x'}{\ud t}\dfrac{1}{\frac{\ud t'}{\ud t}}= \\
	  &= \gamma\left(\frac{\ud x}{\ud t}-u\right)\dfrac{1}{\gamma\left(1-\dfrac{\beta}{c}\dfrac{\ud x}{\ud t}\right)} = \dfrac{v_x-u}{1-\dfrac{\beta}{c}v_x}
\end{split}
\end{equation}
\begin{equation}\label{composizione_vel_y}
\begin{split}
  v'_{y'} &= \frac{\ud y'}{\ud t'}=\frac{\ud y}{\ud t'}=\frac{\ud y}{\ud t}\frac{\ud t}{\ud t'}=v_y\frac{1}{\frac{\ud t'}{\ud t}}= \\
	  &= \dfrac{v_y}{\gamma\left(1-\dfrac{\beta}{c}\dfrac{\ud x}{\ud t}\right)}=\dfrac{v_y\sqrt{1-\beta^2}}{1-\dfrac{\beta}{c}v_x}
\end{split}
\end{equation}
Un'analoga equazione vale per $v'_{z'}$ essendo come $v'_{y'}$ una velocità trasversa rispetto ad $u$.
\[\lim_{\beta\rightarrow 0}v'_{x'}=v_x-u\qquad \lim_{\beta\rightarrow 0} v'_{y'}=v_y\]
cioè la composizione delle velocità secondo Galileo.

\begin{Es}[risolviamo il problema precedente]
Nell'esempio di prima:
\[\begin{array}{ll}
   v_x = 0.8 c\\
   u = - 0.8 c
  \end{array}
\]
applicando la \ref{composizione_vel_x} abbiamo:
\[v'_x=\frac{v_x - u}{1-\frac{\beta}{c}v_x}=\frac{0.8c - (- 0.8c)}{1+\frac{0.8}{c}0.8c} \approx 0.98c\]
ossia $v'_x < c$, come ci aspettavamo.
\end{Es}

\begin{Es}[decadimento $\pi^0$]
$\pi^0$ decade con un decadimento elettromagnetico del tipo:
\[\pi^0\rightarrow \gamma\gamma\]
Per la conservazione della quantità di moto i due fotoni devono avere velocità opposte, $c$ misurate sia dal sistema del laboratorio ($O$) sia dal sistema in moto con il $\pi^0$, $O'$. Supponiamo che i fotoni si propaghino secondo l'asse $x$ e che il mesone si muovesse alla velocità $u$ verso destra rispetto ad $O$. Per Einstein la velocità dei due fotoni vista da $O$ e da $O'$ è uguale.
Per il fotone che va a destra:
\[v_x=\frac{v'_{x'}+u}{1+\frac{\beta}{c}v'_{x'}}=\frac{c+u}{1+\frac{u}{c^2}c}=c\]
Per l'altro:
\[v_x=\frac{-c+u}{1-\frac{u}{c^2}(-c)}=-c\]
\end{Es}

\section{Massa\index{massa!relativistica}}
Dalla conservazione della quantità di moto ricaviamo la massa relativistica.

\begin{Es}[quantità di moto classica vs relativistica]
\begin{figure}[htbp]
   \centering
   \includegraphics[scale=0.5]{immagini/relspec/massa_relat.png}
   \caption{Altri due pistoleri relativistici. I due pistoleri (identici) sono in moto relativo, armati di pistole identiche e con proiettili
identici.}
\end{figure}

Siano $O$ e $O'$ due osservatori dotati di due pistole identiche con proiettili identici. 
$O'$ si muove verso destra rispetto ad $O$ con velocità costante $\ve v$\footnote{Gli assi $x$ e $x'$ non coincidono, sono sfasati, 
ma questo conta poco perché considerando le velocità, cioè derivando, il problema scompare.}. 
$O'$ spara in direzione di $O$ cioè nel verso delle $y$ decrescenti. $O$ vede il suo proiettile con velocità:
\[\dot y'_{O'}=-v\qquad\dot{x'}_{O'}=0\]

Dove abbiamo indicato con un puntino la derivata rispetto al tempo ($\dot y = \dfrac{\partial y'}{\partial t'}$ è quindi una
velocità), con l'apice il fatto che è una quantità misurata dall'osservatore $O'$ e con il pedice che è il proiettile sparato da
$O'$.

$O$ vede il proiettile di $O'$ con velocità:
\[\dot y_{O'}=\frac{\dot y'_{O'}\sqrt{1-\beta^2}}{1-\frac{\beta}{c}\dot x'_{O'}}=-\sqrt{1-\frac{u^2}{c^2}}v\]

$O$ spara a $O'$ nel verso delle $y$ crescenti e misura la velocità del proiettile:
\[\dot y_{O}=v\qquad \dot x_{O}=0\]
e $O'$ vede il proiettile di $O$ con:
\[\dot y'_{O}=\sqrt{1-\frac{u^2}{c^2}}v\]

La quantità di moto si deve conservare e sappiamo che prima che i due pistoleri sparassero era uguale a zero quindi abbiamo che 
l'osservatore $O$ descrive il sistema con la seguete equazione di conservazione della quantità di moto:
\[m_{O}U+m_{O'}\dot y_{O'}=0\]
\[m_{O'}=\frac{m_O v}{\sqrt{1-\frac{u^2}{c^2}}v}=\frac{m_O}{\sqrt{1-\frac{u^2}{c^2}}}\]

Vediamo che entrambi gli osservatori concordano sulla seguente osservazione: se ciascuno degli osservatori indica la massa del proiettile con la quantità $m_0$ (che, per ragioni che saranno evidenti nelle
  prossime righe è detta ``massa a riposo''. Ossia:
\begin{itemize}
 \item $O: m_{O} = m_0$;
 \item $O': m_{O'} = m_0$;
\end{itemize}
la massa del proiettile in moto (indicata semplicemente con $m$) diventa:
\[m=\frac{m_0}{\sqrt{1-\beta^2}}=\gamma m_0\]
Vediamo che la massa di una particella misurata da un'osservatore in moto varia con la velocità del moto relativo e in varticole se $v$ 
diventa prossimo a $c$ abbiamo che:
\[\beta\rightarrow 1\Rightarrow m\rightarrow +\infty\]
\end{Es}

\section{Quantità di moto\index{quantità di moto!relativistica}}
\[p=mv=\frac{m_0}{\sqrt{1-\beta^2}}\beta c=\gamma m_0 v\]
$\beta$ della particella. Con questa nuova definizione di quantità di moto essa si conserva ancora, anche in contesto relativistico.
\[\beta\rightarrow 1\Rightarrow p\rightarrow +\infty\]

\begin{Es}[quantità di moto classica vs relativistica]
\begin{figure}[htbp]
   \centering
   \includegraphics[scale=0.5]{immagini/relspec/Q_rel1}
   \caption{(a) prima dell'urto; (b) dopo l'urto.}
\end{figure}
Urto frontale tra due particelle con stessa massa e velocità, sull'asse $x$. Dopo l'urto le particelle possono andare su qualunque direzione, ipotizziamo quella verticale. La quantità di moto si conserva, infatti all'inizio:
\[p_x=mv_x-mv_x=0\qquad p_y=mv_y-mv_y=0\]
e dopo l'urto:
\[p_x=mv_x-mv_x=0\qquad p_y=mv_y-mv_y=0\]
Quindi sembra che la conservazione della quantità di moto classica funzioni. Cambiamo sistema di riferimento, 
consideriamo quello solidale con la seconda particella. Essa prima dell'urto si muove con velocità $u=v$ verso sinistra. 
Utilizzando le composizioni delle velocità relativistiche:
\begin{figure}[htbp]
   \centering
   \includegraphics[scale=0.5]{immagini/relspec/Q_rel2}
   \caption{(a) prima dell'urto; (b) dopo l'urto.}
\end{figure}

\[\left\{
\begin{array}{l}
p'_{x'}=p'_{x'_1}=\dfrac{v+v}{1+\dfrac{v^2}{c^2}}m=\dfrac{2v}{\dfrac{c^2+v^2}{c^2}}m=2mu\dfrac{1}{1+\beta^2}\\
p'_{y'}=0\\
\end{array}
\right. \]
\[\left\{
\begin{array}{l}
p'_{x'}=2mv'_{x'}=2mu\\
p'_{y'}=0\\
\end{array}
\right. \]
La quantità di moto fallisce. Bisogna considerare la correzione della massa.
\end{Es}
\section{Energia Relativistica\index{energia!relativistica}}
\label{energia_cinetica_relativistica}
\subsection{Teorema lavoro--energia\index{teorema!lavoro--energia!relativistico}}
\begin{figure}[htbp]
   \centering
   \includegraphics[scale=0.6]{immagini/relspec/V_K}
   \caption{Velocità in funzione dell'energia cinetica con diverse masse.}
\end{figure}
Per far funzionare il teorema $L=\Delta K$ bisogna cambiare la definizione di energia cinetica.\index{energia!cinetica!relativistica}
\[m=\frac{m_0}{\sqrt{1-\beta^2}}\]
\[m^2(1-\beta^2)=m_0^2\quad m^2c^2-m^2v^2=m_0^2c^2\]
\[2mc^2\ud m-2mv^2\ud m-2m^2v\ud v=0\]
\[c^2\ud m=v^2\ud m+mv\ud v\]
Considerando solo una forza diretta come l'asse $x$:
\begin{equation}
L= \left(m v_B-m v_A \right)c^2 
\end{equation}

Se $K(0)=0$ in accordo con quella classica:
\[K(B)=K(B)-K(0)=\Delta K=(m_B-m_0)c^2\quad K=mc^2-m_0c^2\]
\[\underbrace{m_0c^2}_{\text{energia a riposo}}+\underbrace{K}_{\text{energia cinetica}}=\underbrace{mc^2}_{\text{energia totale}}\]
\[E=mc^2\qquad \Delta E=\Delta K=L\]
\[K=mc^2-m_0c^2=\frac{m_0c^2}{\sqrt{1-\beta^2}}-m_0c^2=m_0c^2\left(\frac{1}{\sqrt{1-\beta^2}}-1\right)=m_0c^2\left(\gamma-1\right)\]

Per far funzionare il teorema $L=\Delta K$ bisogna cambiare la definizione di energia cinetica.\index{energia!cinetica!relativistica}
\[m=\frac{m_0}{\sqrt{1-\beta^2}}\]
\[m^2(1-\beta^2)=m_0^2\quad m^2c^2-m^2v^2=m_0^2c^2\]
\[2mc^2\ud m-2mv^2\ud m-2m^2v\ud v=0\]
\[c^2\ud m=v^2\ud m+mv\ud v\]
Considerando solo una forza diretta come l'asse $x$:
\[L=\int_A^B \ve F\ud \ve x=\int\frac{\ud \ve p}{\ud t}\ud \ve x=\int\ud \ve p\,\frac{\ud\ve x}{\ud t}=\int \ve v\ud \ve p=\int \ve v\ud(m\ve v)=\Delta K=\]
\[=\!\int(\ud m\ve v+m\ud\ve v)\ve v=\!\!\int\!(\ud mv^2+mv\ud v)=c^2\!\!\!\int_{v_A}^{v_B}\!\!\!\!\ud m=\left(m\left(v_B\right)-m\left(v_A\right)\right)c^2\]
Se $K(0)=0$ in accordo con quella classica:
\[K(B)=K(B)-K(0)=\Delta K=(m_B-m_0)c^2\quad K=mc^2-m_0c^2\]
\[\underbrace{m_0c^2}_{\text{energia a riposo}}+\underbrace{K}_{\text{energia cinetica}}=\underbrace{mc^2}_{\text{energia totale}}\]
\[E=mc^2\qquad \Delta E=\Delta K=L\]
\[K=mc^2-m_0c^2=\frac{m_0c^2}{\sqrt{1-\beta^2}}-m_0c^2=m_0c^2\left(\frac{1}{\sqrt{1-\beta^2}}-1\right)=m_0c^2\left(\gamma-1\right)\]

\subsection{Energia cinetica classica}
L'energia cinetica relativistica si riduce alla formula classica più una costante per per $v \ll c$:
\[v\ll c\qquad m_0c^2\left(1+\frac{1}{2}\beta^2+\ldots-1\right)=\frac{1}{2}m_0\beta^2c^2=\frac{1}{2}m_0v^2\]

\subsection{Urti ed energia}
\[E=m_0c^2+K=mc^2\]

\begin{figure}[htbp]
   \centering
   \includegraphics[scale=0.5]{immagini/relspec/urto_rel.png}
   \caption{(a) prima dell'urto; (b) dopo l'urto. L'energia si conserva, ma viene ``ripartita'' diversamente dato che si sono formate
nuove particelle.}
\end{figure}

Si creano nuove particelle. Non basta avere molta energia per creare nuove particelle, ma bisogna concentrarla.

\section{Il triangolo relativistico\index{triangolo!relativistico}}
Partiamo dalle seguenti:
\begin{equation}
\left\{
\begin{array}{l}
E=mc^2=\dfrac{m_0}{\sqrt{1-\beta^2}}c^2\\
p=\dfrac{m_0}{\sqrt{1-\beta^2}}\beta c\\
\end{array}\right.
\quad\Rightarrow\quad
\dfrac{p}{E}=\dfrac{\beta}{c}
\quad\Rightarrow\quad\beta E=pc
\end{equation}
con qualche passaggio abbiamo che:
\[E^2=\frac{m_0^2c^4}{1-\beta^2}\qquad E^2-\beta^2 E^2=m_0^2c^4\qquad E^2-p^2c^2=m_0^2c^4\]
Alla fine possiamo scrivere che:
\begin{equation}
 E^2=(m_0c^2)^2+(pc)^2
\end{equation}

Tutte le relazioni di cui sopra possono essere facilmente ricordate con il cosiddetto ``triangolo relativistico'':
\begin{figure}[htbp]
   \centering
   \includegraphics[scale=0.4]{immagini/relspec/Trg_rel.png}
   \caption{Il ``triangolo relativistico''. È un utile strumento mnemonico, infatti, tra le grandezze indicate sui lati esistono relazioni 
analoghe a quelle tra i lati di un triangolo rettangolo.}
\end{figure}

\subsubsection{Elettronvolt\index{elettronvolt}}
La carica dell'elettrone vale circa $\SI{1.6E-19}{\coulomb}$. $\SI{1}{\electronvolt}$ è quell'energia che acquista un elettrone accelerato da una ddp di $\SI{1}{\volt}$
\begin{equation}\label{eV_to_Joule}
E=qV\qquad \electronvolt=q_e(\SI{1}{\volt})=\SI{1.6E-19}{\joule} 
\end{equation}
Molto usati sono i multipli: \kilo\electronvolt (si legge ``\textit{chev}''), \mega\electronvolt (``\textit{mev}''), 
\giga\electronvolt (``\textit{gev}''), \tera\electronvolt (``\textit{tev}''), \peta\electronvolt (``\textit{pev}'').

Anche le masse e le quantità di moto si possono esprimere in \electronvolt, combinando opportunamente con $c$.
Sappiamo che:
\[E=mc^2\]
Quindi possiamo trovare, per esempio, le seguenti equivalenze:
\begin{align}
\SI{1}{\kilo\gram}c^2 &=\SI{1}{\kilo\gram}(\SI{3E8}{\metre\per\second})^2=\SI{9E16}{\joule}=\SI{5.62E35}{\electronvolt} \\
\SI{1}{\electronvolt}/c^2 &=\SI{1.78E-36}{\kilogram}
\end{align}
Mentre la massa di un elettrone espessa in \electronvolt è data da:
\[E_0 = m_0 c^2 \rightarrow m_0 = \dfrac{E_0}{c^2}\]
quindi, sapendo che la massa a riposo dell'elettrone è $m_{e} = m_{0} = \SI{9.1E-31}{\kilo\gram}$ ed usando il fattore di 
conversione \ref{eV_to_Joule} abbiamo:
\begin{equation}
\begin{split}
  m_{e}c^2 &= (\SI{9,1E-31}{\kilo\gram})(\SI{3E8}{\metre\per\second})^2 = \dfrac{\SI{8.19E-14}{\joule}}{\SI{1.6E-19}{\joule\per\electronvolt}} = \\
	   &= \SI{5.11E5}{\electronvolt} = \SI{511}{\kilo\electronvolt}
\end{split}
\end{equation}

Quindi la massa di un elettrone è $m_e = \SI{511}{\frac{\kilo\electronvolt}{c^2}}$. Dato che $c$ è una costante spesso viene omessa (o posta
uguale ad un'unità, vedi la sez. \ref{analisi-dimensionale-unita-naturali}, e allora si dice semplicemente che la massa di un elettrone
è $\SI{511}{\kilo\electronvolt}$.

\begin{Es}[decadimento particella strana]
Si vuole determinare il momento finale dei prodotti del decadimento della $K_0$:
\[K_0\rightarrow \pi^+\,\pi^-\]
\[m_0^{K_0}=\SI[per=slash,eVcorrb=0.4ex]{500}{\mega\eVperc\squared} \qquad m_0^{\pi^+}=\SI{140}{\mega\eVperc\squared}\]
\[\tau_{K_0}=\SI{0.9E-10}{\second} \]
Nel sistema di $K_0$ l'impulso iniziale è nullo e l'impulso finale dei mesoni $\pi^{\pm}$ è:
\[
\ve p_i = p_{K_0}=0\quad \Rightarrow \quad \ve p_f=0 = \ve p_{\pi^+}+\ve p_{\pi^-}
\]
Poniamo:
\[
m_{0}^{\pi^{+}}=m_0^{\pi^{-}}=m_{\pi} \qquad m_0^{K_0}=m_{K}
\]
Inoltre sappiamo che il momento finale dei due mesoni è uguale in modulo:
\[
p_{\pi^+}=p_{\pi^-}=p_{\pi}\qquad E_i=E_f
\]

Quindi l'energia è data da:
\begin{equation*}
 \begin{split}
   E_i &= m_{K}c^2 = E_f = \\
       &= \sqrt{(m_0^{\pi^+}c^2)^2+p_{\pi^+}^2c^2}+\sqrt{(m_0^{\pi^-}c^2)^2+p_{\pi^-}^2c^2} = \\
       &= 2 \sqrt{(m_\pi c^2)^2+p_{\pi}^2c^2}
 \end{split}
\end{equation*}

da cui:
\[m_{K}^2c^4=4(m_\pi c^2)^2+4p_{\pi}^2c^2\]
\[p_\pi^2=\frac{m_{K}^2c^4-4(m_{\pi})^2c^4}{4c^2}=\frac{m_K^2c^2}{4}-m_\pi^2c^2\]
\[p_\pi=c\sqrt{\frac{m_K^2}{4}-m_\pi^2}\]
\end{Es}

\section{Analisi dimensionale ed unità naturali}\label{analisi-dimensionale-unita-naturali}
In base a quanto detto finora vediamo che possiamo stabilire una certa simmetria tra le misure 
temporali e quelle spaziali, infatti anche nelle trasformazioni di Lorentz compare sempre la quantità
$ct$ che ha le stesse dimensioni di $x$.

Come detto sopra possiamo porre la costante $c=1$, proprio per indicare che essa è l'unità di misura rispetto alla
quale misuriamo tutte le velocità.

Se ora sottoponiamo le grandezze incontrate finora ad un'analisi dimensionale possiamo trovare conferma di questi
parallelismi.

Nell'analisi dimensionale ad ogni grandezza fisica si associa un numero (intero) naturale che consiste nell'esponente
al quale viene elevata una delle grandezze fondamentali (in meccanica: lunghezza, tempo e massa) che lo compongono.
Questi esponenti vengono detti ``dimensioni'' della grandezza sotto esame rispetto alla lunghezza, tempo, ecc.

Si indicano le dimensioni di una grandezza usando le parentesi quadre $[...]$, per le grandezze fondamentali si usa:
\begin{itemize}
 \item $[L]$ per le lunghezze
 \item $[T]$ per i tempi
 \item $[M]$ per le masse
\end{itemize}
si può applicare questo discorso anche alle unità di misura.

Ad esempio:
\begin{itemize}
 \item le dimensioni di una forza sono date da $[F] = [M][L][T]^{-2}$;
 \item le dimensioni della costante elastica di una molla sono $[k] = [F][L]^{-1} = [M][T]^{-2}$;
\end{itemize}

Molto spesso ricorrendo alla analisi dimensionale è possibile ricavare la soluzione di un problema o almeno un suggerimento verso di essa.

Per esempio, dato il problema seguente:
\begin{quotation}
 Dato un pendolo composta da un pesetto di massa $m$ legato ad un filo di lunghezza $l$ e sottoposto alla forza di gravità 
(accelerazione di gravità $g$), trovare il periodo del pendolo.
\end{quotation}
Applicando l'analisi dimensionale sappiamo che stiamo cercando una grandezza, il periodo $T_p$, le cui dimensioni sono quelle di 
un tempo: $[T_p] = [T]$.
I dati che abbiamo a disposizione sono la massa $m$, la lunghezza del pendolo $l$ e l'accelerazione di gravità $g$ le cui dimensioni
sono rispettivamente $[m]=[M]$, $[l] = [L]$, $[g] = [L][T]^{-2}$

Quindi possiamo impostare la seguente equazione:
\begin{equation}
 T_p \propto m^{\alpha} l^{\beta} g^{\gamma}
\end{equation}
ovvero, dimensionalmente:
\begin{equation}
 [T_p] = [m]^{\alpha} [l]^{\beta} [g]^{\gamma}
\end{equation}

e quindi troviamo le seguenti equazioni dimensionali:
\begin{equation}
\left\{
\begin{aligned}
 0 &= \alpha & & ([M])\\
 0 &= \beta + \gamma & & ([L])\\
 1 &= -2 \gamma & & ([T])
\end{aligned}
\right.
\end{equation}
che risolto restituisce $\alpha = 0$, $\beta = \frac{1}{2}$, $\gamma = - \frac{1}{2}$, ossia
\begin{equation}
 T_p \propto \sqrt{\frac{l}{g}}
\end{equation}
la soluzione esatta è data da:
\begin{equation}
 T_p = 2 \pi \sqrt{\frac{l}{g}}
\end{equation}
e si vede che la costante di proporzionalità (adimensionale) è data da $2 \pi$

In relatività, lo spazio ed il tempo diventano grandezze derivate dalla grandezza fondamentale ``velocità'' quindi abbiamo:
\begin{equation}
 [c] = \dfrac{[L]}{[T]} = 0 \rightarrow [L] = [T]\end{equation}

Inoltre dalla relazione $E = mc^2$ ricaviamo che:
\begin{equation}
 [E] = [M]\\
\end{equation}

Se facciamo uso della costante di Planck, ovvero della costante fondamentale della meccanica quantistica, 
pari a:
$h = 6,626 \times 10^{-34} J s = 4,136 \times 10^{-15} eV s$
e stabiamo che anch'essa sia $h = 1$\footnote{molto spesso si usa invece di $h$ la quantita $\hbar$ (``acca tagliato'') detta
costante di Dirac o costante di Plack ridotta pari a:
\[\hbar = \dfrac{h}{2 \pi} \]
e si pone $\hbar = 1$. Dal punto di vista dimensionale queste due condizioni sono identiche.}
possiamo stabilire un'altra equivalenza tra grandezze fisiche ovvero, sapendo che 

\begin{equation}
 [h] = [E][T] = 0 \rightarrow [E] = [M] = [T]^{-1} = [L]^{-1}
\end{equation}

Quindi per l'analisi dimensionale valgono le seguenti uguaglianze:
\begin{equation}
 \left\{
 \begin{aligned}
  [L] &= 1\\
  [T] &= 1\\
  [M] &= -1\\
  [E] &= -1
 \end{aligned}\right.
\end{equation}



\section{Un nuovo invariante relativistico}

Giunti a questo punto, e ricordando come abbiamo utilizzato i postulati della relatività per ottenere
le trasformazioni di Lorentz, possiamo verificare l'esistenza di una nuova quantità invariante:
\begin{equation}\label{ds2}
 ds^2 = dx^2 + dy^2 + dz^2 - ct^2
\end{equation}

$ds^2$ in \ref{ds2} è noto come ``intervallo spaziotemporale'' ed è una quantità molto importante che definisce
numerose proprietà geometriche dello spaziotempo tra cui la sua metrica e la sua struttura causale.

Per ora ci limitiamo ad osservare due fatti: (1) la \ref{ds2} assomiglia alla generalizza a quattro dimensioni
della formula per la lunghezza di un vettore nello spazio:
\[ l^2 = x^2 + y^2 + z^2 \]
con la differenza che il tempo ha un segno meno davanti, questa è una differenza non da poco, in un certo senso possiamo
dire che è ciò che distingue le coordinate spaziali da quelle temporali.

Inoltre, se con i vettori nello spazio ``usuale'' ($\field{R}^3$ euclideo) la lunghezza era una quantità sempre positiva,
abbiamo che \ref{ds2} può essere anche negativa o nulla, in particolare si dice che un vettore nello spaziotempo individuato
dalle sue componenti spaziotemporali $(x,y,z,t)$ è:
\begin{itemize}
 \item di tipo spazio (in inglese \textit{space-like}), se $ds^2 > 0$
 \item di tipo tempo (in inglese \textit{time-like}), se $ds^2 < 0$
 \item di tipo luce (in inglese \textit{light-like}), se $ds^2 = 0$
\end{itemize}

Vediamo infatti che un raggio di luce occupa sempre punti tali che:
\[ x^2 + y^2 + z^2 = ct^2 \]
infatti si propaga con velocità $c$, quindi:
\[ ds^2 = 0 \]

Queste distinzioni acquisteranno ancora più significato quando parleremo della \textit{struttura causale} dello spaziotempo.