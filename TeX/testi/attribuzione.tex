\chapter[Credits]{Credits}

Questo mini-corso è stato costruito \sout{copiando spudoratamente} prendendo ispirazione da due fonti principali:
\begin{itemize}
 \item le dispense preparate da R. Turra a partire dalle lezioni di Fisica 1, 2, 3 seguite all'università di Milano-Bicocca.
 \item le ``Dodici lezioni sulla relatività'' del prof. F. Magri, scritte, integrate e organizzate da D. Ghisi
\end{itemize}

Ovviamente, ``ça va sans dire'', durante il lavoro di copia-incolla-taglia-cuci probabilmente ho introdotto numerosi errori e i cambiamenti di 
stile e di notazione tra le varie parti del testo dovrebbero essere oltremodo evidenti. Ovviamente mi prendo la responsabilità di tutto questo
e mi dispiace di avere senza dubbio rovinato la compattezza delle dispense di Fisica e 
l'eleganza delle lezioni di Relatività.

Vi prego infine di ricordare che queste dispense sono assolutamente una bozza preliminare, numerosi sono i punti da chiarire e da integrare
per rendere questo testo un vero ``Mini-corso sulla Relatività''.

Potete contattarmi scrivendo all'indirizzo:
\begin{center}
\framebox[1.1\width][c]{\textup{\textsf{\href{mailto://kikkocristian@gmail.com}{kikkocristian@gmail.com}}}}
\end{center}
\rmfamily\upshape

Infine, è doveroso ricordare che, a suo tempo, ho preparato degli esami studiando sulle dispense citate (in particolare ``Fisica 1 - mod I e II'' e ``Istituzioni 
di Fisica Matematica - mod. I'') incassando ottimi voti, colgo quindi l'occasione per ringraziare gli autori originali anche per questo. 

Sono riportati di seguito dei riferimenti a questi lavori.

\begin{itemize}
 \item Le dispense di R. Turra, riportavano il seguente disclaimer
\end{itemize}

\begin{quote}
\section*{\centering Disclaimer}
%\addstarredsection{Disclamer} % per minitoc
\addcontentsline{toc}{section}{\numberline{}Disclaimer}
Di questo documento puoi farne quello che vuoi, distribuirlo, fotocopiarlo, usare come carta per accendere il camino 
o per non sporcare per terra quando imbianchi i muri, a patto che l'autore originale 
e la sua email vengano riportati in modo significativo insieme a questo disclamer nella sua forma originale. 
Possono essere fatte modifiche solo allo scopo di migliorare il do\-cu\-men\-to. 
L'autore sarà felice di ricevere una copia di queste modifiche. 
In nessun caso questo documento, parti di esso o opere derivate potranno essere utilizzati come  fonte di lucro; 
l'unico eventuale ricavo ammesso è quello relativo alle spese di distribuzione, per esempio le spese di stampa. 
Questo documento non è la Bibbia, esso nasce per uso personale, ed è distribuito senza garanzia sui contenuti 
(se prendente un brutto voto studiando su questi appunti non prendetevela con me).

L'autore può essere contattato all'indirizzo:
\begin{center}
\framebox[1.1\width][c]{\textup{\textsf{\href{mailto://giurrero@gmail.com}{giurrero@gmail.com}}}}
\end{center}
\end{quote}
\rmfamily\upshape

La versione utilizzata per preparare il ``Mini-corso sulla Relatività'' è quella di aprile 2011 (Revision: $120$). 
I sorgenti sono disponibili all'indirizzo \href{http://code.google.com/p/fisica123}{http://code.google.com/p/fisica123}.

\begin{itemize}
 \item le dispense di D. Ghisi sono disponili all'indirizzo: 
\end{itemize}

\begin{center}
\framebox[1.1\width][c]{\textup{\textsf{\href{www.webalice.it/dghisi/scritti/relativita.pdf}{www.webalice.it/dghisi/scritti/relativita.pdf}}}}
\end{center}
\rmfamily\upshape

\section[Licenza]{Licenza}

Questo lavoro è disponibile secondo la licenza \textbf{Creative Commons - Attribuzione - Non commerciale - Condividi allo stesso modo}
(CC-BY-NC-SA) versione 3.0.

In sostanza è possibile riprodurre, distribuire, mettere a disposizione per lo scaricamento, 
comunicare al pubblico, esporre in pubblico, rappresentare, eseguire e recitare quest'opera
o modificarla a condizione di attribuire la paternità dell'opera riportando la seguente indicazione:
\begin{quote}

\textup{\textsf{Tratto da ``Un mini-corso sulla Relatività di Einstein''- di Cristian Consonni (da lavori precedenti di R. Turra e D. Ghisi)}}

\textup{\textsf{È possibile contattare l'autore orginale all'indirizzo kikkocristian@gmail.com.}}
\end{quote}

È inoltre necessario riportare il \textbf{Disclaimer} indicato da R. Turra e riportato sopra.
L'attribuzione di quest'opera in lavori derivati deve essere inserita in modo da non suggerire che gli autori originali 
avallino le opere derivate stesse o il modo in cui la loro opera è utilizzata.

Non è possibile quest'opera per fini commerciali, fatte salve le spese di distribuzione (ad es. rimborso spese di stampa).

Se si altera o trasforma quest'opera, o se la si usa per crearne un'altra, è possibile distribuire l'opera risultante solo 
con una licenza identica o equivalente a questa. 

Ulteriori informazioni ed il testo completo della licenza CC-BY-NC-SA sono disponibili alla 
pagina \url{http://creativecommons.org/licenses/by-nc-sa/3.0/deed.it}
