http://it.wikipedia.org/wiki/File:Interferometro-Michelson.png

Principio della relatività: le leggi della fisica sono le
stesse in qualsiasi sistenza di riferinzento inerzia!e.
Principio della costanza della velocità della luce: la
velocità del/a luce ha un identico l}alore in qualsiasi
sistelna di riferùnento inerziale.
TI primo principio afferma che le leggi della fisica sono as-
solute, universali e identiche per tutti gli osservatori iner-
ziali. Una legge valida per un osservatore inerziale non può
essere violata nei confronti di alcun altro osservatore iner-
ziale.
Il secondo postulato è molto più difficile da accettare,
perché urta contro il nostro senso comune,' fermamente ra-
dicato nella cinematica galileiana, che è sempre stata con-
fermata dalle nostre esperienze quotidiane. Consideriamo
tre osservatori, 040, B e C, ciascuno dei quali sta fermo in un
diverso sistema di riferimento inerziale. Un lampo di luce
viene emesso da .A, il quale rileva che la luce viaggia alla
velocità c. Il sistema inerziale B si a/lontana da A alla velo-
cità c/4: la cinematica di Galileo prevede che B misuri una


La Relatività è lo studio della geometria dello spaziotempo, cioè
lo studio delle strutture di questo spaziotempo che non dipendono dalle modalità
dei processi di misura. I processi di misura dipendono infatti dall’osservatore 
che li esegue: l'obiettivo della Relatività è dunque individuare le caratteristiche che si presentano 
indipendentemente dall'osservatore e dai processi di misura. Le proprietà manifestate in
comune dalle misure eseguite dai possibili osservatori vengono interpretate come proprietà dello spaziotempo 
e ne definiscono la struttura.

Lo spaziotempo è l’insieme degli eventi, cioè dei fenomeni localizzati nello spazio 
(a cui si può attribuire una localizzazione spaziale) e di brevissima durata 

Un osservatore è quindi un sistema di coordinate sullo spaziotempo. Gli osservatori sono definiti operativamente
 dalle modalità dei processi di misura che determinano le coordinate dell’evento. La nozione di osservatore
ha visto un'ampia evoluzione nel corso del tempo; tale evoluzione è sostanzialmente andata di pari 
passo con un'evoluzione della geometria. Il cambiamento sostanziale nella concezione di osservatore che 
si è verificato con il passaggio dalla Relatività newtoniana alla Relatività di Einstein è associato
al passaggio dalla geometria euclidea alla geometria di Riemann.

