\chapter{Relatività generale}

Torniamo per un attimo ad assumere un punto di vista cronologico. Una
volta messa a punto la geometria dello spazio di Minkowski (cioè della relatività 
ristretta), si aprivano di fronte a Einstein due problemi a prima vista
scorrelati.

\begin{enumerate}
 \item[i)] In primo luogo, tutta la teoria relativistica svolta sino ad ora ha utilizzato 
il concetto di osservatore inerziale ed ha avuto come scopo la
scrittura delle equazioni della fisica in una forma indipendente dalla
scelta dell'osservatore inerziale, al fine di mostrare la compatibilità di
tali equazioni con il Principio di Relatività. 
Il problema di Einstein nel 1906 è dunque quello di estendere la teoria precedente, così da
conglobare in essa anche osservatori non inerziali, con il proposito di
scrivere le equazioni della fisica in una forma che sia invariante nel passaggio 
da un osservatore a un altro. Lo scopo è quindi il superamento del concetto di osservatore inerziale e della cosiddetta 
formulazione covariante\footnote{Cioè formulazione indipendente dal sistema di riferimento inerziale scelto. Nella sua
forma estesa, la formulazione indipendente dall'osservatore (qualsiasi) scelto verrà detta ``formulazione covariante generale''}
delle equazioni della fisica.
È dunque in questo senso che questa parte della teoria è stata denominata da 
Einstein Relatività Generale (dal momento che considera la classe generale 
degli osservatori, pensati sempre come sistemi di coordinate sullo spaziotempo), per contrasto con la parte della teoria basata
sull'uso dei soli osservatori inerziali (detta Relatività Ristretta)\footnote{Va 
sfatato un deleterio luogo comune: si legge e si sente dire spesso che la Relatività
Ristretta si applica solo ai sistemi di riferimento inerziali, cioè al caso di moti non accelerati. 
Non c'è nulla di più falso: la Relatività Ristretta va bene per qualsiasi tipo di moto, nell'ipotesi che 
lo spaziotempo sia piatto, ossia nell'ipotesi di assenza di gravità. Addirittura, 
la Relatività Ristretta nasce dalle simmetrie dell'elettromagnetismo, e dunque non
ha alcun problema a contemplare (ad esempio) elettroni accelerati.}.

 \item[ii)] D'altro canto, il proposito della Relatività Ristretta di scrivere le equazioni 
della fisica in forma invariante rispetto al gruppo di Lorentz (cioè
rispetto al cambio di osservatore inerziale) doveva ancora essere completato. 
Dopo aver realizzato questo scopo per la meccanica della particella, per la meccanica dei fluidi 
e per 
l'elettromagnetismo\footnote{Ricordiamo che, se per rendere compatibili le equazioni della meccanica con il Principio
di Relatività abbiamo dovuto modificare le equazioni di Newton, le equazioni dell'elettromagnetismo erano 
già compatibili con tale principio, e quindi invarianti per trasformazioni di Lorentz.},
rimaneva da trattare la gravitazione. E ovvio che la gravitazione newtoniana, 
basata sul concetto di simultaneità assoluta e di azione a distanza istantanea, non è compatibile con i principi 
della Relatività Ristretta\footnote{Difatti, l'asserto newtoniano secondo cui due masse a distanza $r$ esercitano una forza
diretta lungo la congiungente in verso centripeto e proporzionale a $r^{-2}$ , presuppone che
spostando una delle due masse l'informazione arrivi all'altra massa istantaneamente, e
dunque che la forza vari istantaneamente (concetto di azione a distanza istantanea). Inol-
tre per avere la distanza tra le due particelle dobbiamo compiere le misurazioni in uno
stesso istante, il che implica che deve esserci simultaneità assoluta.};
si tratta quindi di riaffrontare il problema.
\end{enumerate}

La prima idea è che bisogna procedere con le equazioni di Newton del 
campo gravitazionale così come abbiamo fatto per le equazioni newtoniane 
della dinamica, cioè modificandole (senza stravolgerle) per renderle compatibili 
con la geometria dello spaziotempo di Minkowski, di cui lo spaziotempo newtoniano 
non è altro che un'approssimazione locale. Si può qualificare questo secondo problema 
come il problema di conglobare la teoria della gravitazione nello
spaziotempo di Minkowski.

Tuttavia si scopre che la gravitazione comporta la modifica della struttura
dello spaziotempo di Minkowski. Il fulcro è dunque che la teoria della gravitazione 
si identifica con la geometria dello spaziotempo, il quale acquista un carattere dinamico.

Dal punto di vista della prospettiva storica attuale, i due problemi, che
per Einstein avevano uguale importanza, svolgono un ruolo completamente
diverso. I problemi della Relatività Generale e del principio di covarianza
generale\footnote{Cioè il principio secondo cui le leggi fisiche devono essere le stesse per ogni sistema
di riferimento. Si noti che da tale principio consegue quasi immediatamente che lo strumento 
matematico per la descrizione della Relatività Generale non può che essere il calcolo
tensoriale.} - punto (i) - hanno poi assunto un carattere via via più secondario.
Nella prospettiva moderna, la teoria della Relatività Generale è definita
come la teoria della gravitazione einsteiniana, identificata con lo studio della
geometria dello spaziotempo - punto (ii).

\section{Forze apparenti}

Passando da un osservatore inerziale ad un osservatore non inerziale
bisogna introdurre le cosiddette forze apparenti. 

Mettiamoci infatti nell'ottica dello schema newtoniano (l'unico che al riguardo per ora conosciamo); 
il primo carattere delle forze apparenti è di essere strettamente proporzionali alla massa inerziale della particella 
su cui si esercitano\footnote{
Ad esempio la forza di Coriolis è data da:
\[F_{Coriolis} = -2m \omega  \wedge v r\]}.

Il secondo carattere delle forze apparenti è (appunto!) di essere apparenti:
il che, per definizione, significa che esse possono essere rigorosamente e globalmente 
annullate semplicemente cambiando osservatore, cioè passando dall'osservatore di cui si sta studiando 
il moto a un cert'altro osservatore. Sappiamo che quest'altro osservatore sarà accelerato rispetto al primo.

Questa caratteristica è detta proprietà di annullamento globale delle forze apparenti: basta cambiare osservatore, 
e rispetto a questo nuovo osservatore immediatamente tutte le forze apparenti scompaiono.

\subsection{Caratteri della forza gravitazionale}
La prima proprietà delle forze apparenti (proporzionalità alla massa iner\-ziale $m_i$) è
 rigorosamente condivisa dalla forza di gravitazione: è noto infatti che tutte le particelle, 
in un campo gravitazionale uniforme e stazionario, ca\-dono con la stessa accelerazione $g$, 
detta accelerazione di gravità. 

Quindi, per la legge di Newton
\begin{equation}
F_{grav} = m_{i} · g
\end{equation}
dove $m_i$ è la massa inerziale. In questo modo si ottiene che:
\begin{equation}\label{massa_inerz}
m_i · a = F_{grav} = m_i · g
\end{equation}
da cui
\begin{equation}
a = g
\end{equation}
e quindi tutte le particelle cadono con uguale accelerazione.
Per sottolineare questo fatto si usa introdurre una ``carica gravitazionale''
$m_g$, detta massa gravitazionale, e scrivere che:
\begin{equation}
F_{grav} = m_g \cdot g,
\end{equation}
in analogia con la formula della forza elettrica:
\begin{equation}
F_{elettr} = q \cdot E
\end{equation}

Possiamo quindi riscrivere l'equazione \ref{massa_inerz} come
\begin{equation}
m_i \cdot a = m_g \cdot g
\end{equation}
e la proprietà scoperta da Galileo dell'universalità dell'accelerazione di gra\-
vità si esprime quindi dicendo che:
\begin{equation}\label{principio_equivalenza_debole}
m_i = m_g 
\end{equation}
La \ref{principio_equivalenza_debole} è detta principio di equivalenza debole: lo assumeremo
come un postulato\footnote{Il più famoso esperimento che lo avvalora è l’esperimento di Eotvos, di bilanciamento
tra forze gravitazionali e forze apparenti (per approfondimenti: \url{http://it.wikipedia.org/wiki/Esperienza_di_Eotvos}.};
esso ci garantisce che (per qualche strana ragione che può apparire quasi ``magica'') massa gravitazionale e massa inerziale
coincidono sempre.
Anche la seconda proprietà delle forze apparenti (la proprietà di an\-nullamento, che giustifica il loro nome di ``forze apparenti'')
è condivisa dalle forze gravitazionali, ma con un'importante differenza: essa vale ora solo localmente.

\section{Osservatori in caduta libera}
Per comprendere questo fatto, consideriamo anzitutto il caso speciale di
un campo gravitazionale uniforme (costante) nello spazio e stazionario nel
tempo; quindi $\vec{g}$ è un vettore costante nello spaziotempo rispetto ad un opportuno 
osservatore inerziale. Proprio per il principio di equivalenza debole,
possiamo annullare il campo gravitazionale in ogni punto dello spazio semplicemente 
passando ad un osservatore in caduta libera, cioè un osservatore
non inerziale, che si muove rispetto all'osservatore inerziale precedente con
un'accelerazione $a = g$. 
Rispetto a tale osservatore, secondo le leggi della
dinamica di Newton, una certa particella di prova si muove sotto l'azione
combinata della gravitazione (dovuta al fatto che essa è in un campo gravitazionale) 
e della forza apparente di gravitazione (dovuta al fatto che stiamo
considerando un osservatore non inerziale). Dunque l'equazione di moto è
$m_i a_{rel} = F_{grav} + F_{app} = m_g g + mi (-aapp ) = mg g - mi aapp = 0$,
dove $arel$ è l'accelerazione totale della particella relativa all'osservatore in caduta 
libera, $F_grav$ è la forza gravitazionale agente sulla particella e $F_app$ è la
forza apparente (detta anche forza di trascinamento) che agisce sulla parti-
cella in virtù del fatto che siamo in un riferimento non inerziale. Abbiamo
usato nell'ultimo passaggio il principio di equivalenza debole ($m_g = m_i$) e
il fatto che, per l'osservatore in caduta libera, l'accelerazione apparente di
trascinamento eguaglia in modulo esattamente l'accelerazione gravitazionale,
ed è di verso opposto a questa. Perciò nel campo gravitazionale costante e
stazionario la particella si muove, rispetto all'osservatore in caduta libera,
come se fosse libera, come se non fosse soggetta ad alcuna forza.
Poiché ogni campo gravitazionale può essere localmente\footnote{
Qui ``localmente'' deve essere inteso in senso spaziotemporale, 
cioè per ``piccole'' regioni di spazio, così piccole da poter considerare $\vec{g}$ 
uniforme in tale regione) e per ``brevi'' intervalli di tempo,
così brevi che la particella di prova non esca, in questo intervallo di
tempo, dalla regione in cui $g$ può essere considerato uniforme.} considerato costante, 
ne segue che la proprietà di annullamento, globale per ogni campo
uniforme e stazionario, vale localmente per ogni campo gravitazionale\footnote{È 
questa la situazione di ``moto in assenza di peso'' che si verifica, ad esempio, in ogni
navicella spaziale orbitante attorno alla terra: in assenza di spinta dei motori, essa si trova
infatti in caduta libera rispetto alla terra (``cade'' infatti sulla terra con un'accelerazione
centripeta g), e dunque la gravità terrestre è localmente annullata.}.

\subsection{Origine degli osservatori inerziali}
L'aver collegato il concetto di osservatore (localmente) inerziale alla geometria 
dello spaziotempo, e quindi in senso fisico alla geometria dell'universo,
ha anche un aspetto soddisfacente da un punto di vista fisico. Un problema
generale che ha assillato i fisici\footnote{In effetti, questo è un punto critico assai problematico 
per la fisica in generale.},
da Newton in avanti, era di spiegare da che cosa fossero determinati gli osservatori 
inerziali, perché cioè un certo osservatore fosse inerziale, mentre un altro non lo fosse. 
Newton rispondeva a
questa domanda con l'assioma dello spazio assoluto: gli osservatori inerziali
sono dunque quelli in quiete rispetto allo spazio. Questa giustificazione non
può più funzionare per Einstein.

La verifica sperimentale che nel nostro sistema solare gli osservatori iner-
ziali fossero in qualche modo legati alla materia distante (al ``cielo delle stelle
fisse'') aveva portato Mach a sviluppare l'idea che gli osservatori inerziali 
dovessero essere definiti dalla distribuzione di materia nell'universo\footnote{
In sostanza, Mach ritiene che se ci fosse un solo corpo nello spazio vuoto non avrebbe
nemmeno senso la nozione di osservatore inerziale.}. 

Gli osservatori inerziali sarebbero quindi corpi in una particolare relazione con 
la materia distante (principio o punto di vista di Mach).

Einstein riprende il punto di vista di Mach e lo realizza compiutamente:
comprendendo che la materia fissa la geometria dello spaziotempo e sapendo
che la geometria fissa gli osservatori localmente inerziali, egli realizza con-
cretamente il legame tra materia distante\footnote{
Perché ``distante''? Perché per fissare la curvatura dello spaziotempo in un punto non
è sufficiente la materia nel punto: servono le derivate prime dei simboli di Christoffel, cioè
le derivate seconde della metrica, e dunque entra in gioco la materia distante.} 
e osservatori inerziali. Quindi, l'idea degli osservatori in caduta libera realizza il principio di Mach.

\subsection{Il principio di equivalenza forte}
Una volta arrivati a questo concetto è naturale assimilare gli osservatori in
caduta libera agli osservatori inerziali dello spaziotempo di Minkowski, accettando 
che nei riferimenti degli osservatori in caduta libera i fenomeni fisici
avvengano localmente tutti con le stesse modalità (a parità di condizioni
ambientali) e che le leggi della fisica assumano in tali riferimenti la forma
minkowskiana. Questa affermazione prende il nome di principio di equivalenza forte: 
esso postula la completa equivalenza (locale) degli osservatori
in caduta libera con gli osservatori inerziali di minkowskiana memoria\footnote{
In altre parole, postula che localmente gravità e accelerazione siano indistinguibili.
Naturalmente questo vale solo localmente: se lasciamo cadere due palline in un razzo in accelerazione,
tali palline seguiranno sempre traiettorie parallele, mentre se lasciamo cadere
le stesse palline da posizioni molto diverse della superficie terrestre, entrambe si indirizzeranno 
verso il centro della terra (con evidente rottura della simmetria tra accelerazioni
apparenti e gravità).}.

Nello stesso ordine di idee è naturale assimilare le particelle in caduta
libera (cioè non soggette a forze di natura diversa dalla forza gravitazionale)
alle particelle libere tout-court di Newton, e quindi postulare che le linee di
universo di tali particelle siano geodetiche del genere tempo nello spaziotempo
(principio della geodetica). Tornando all'analogia con il trampolino di
figura 10.3, un oggetto di prova (ad esempio: una pallina di ping-pong)
posizionato sul trampolino incurvato, accelererà verso la sfera centrale in
un modo governato dalla curvatura stessa del trampolino. Il principio della
geodetica ci dice come la curvatura dello spaziotempo (cioè del trampolino)
governa il moto della materia: i corpi in caduta libera (come la pallina da
ping-pong) seguono geodetiche nello spaziotempo. Ad esempio, inviando la
pallina con la giusta velocità e direzione, essa inizierà a orbitare intorno alla
palla di marmo centrale (come la luna orbita intorno alla terra).
Concettualmente, abbiamo sostituito all'idea di forza gravitazionale la
geometria: la traiettoria della pallina da ping-pong non è più dovuta alla
gravità, piuttosto diciamo che è dovuta alla curvatura dello spaziotempo e
al principio che vuole la traiettoria della pallina essere una geodetica in tale
spaziotempo. In generale stiamo considerando fenomeni che classicamente
erano legati all'azione della gravità (come la caduta dei gravi o il moto di
una navicella orbitante) come fenomeni di caduta libera, che localmente sono
a tutti gli effetti inerziali\footnote{
Così ad esempio, quando percepiamo, stando sulla superficie terrestre, una ``forza di
gravità'', essa è sostanzialmente un risultato del fatto che, non essendo noi osservatori
in caduta libera, siamo continuamente sottoposti a un'accelerazione fisica causata dalla
resistenza meccanica della superficie terrestre.}.

La presenza di un campo gravitazionale non uniforme, che quindi non può
essere annullato globalmente e che implica la curvatura dello spaziotempo, si
manifesta nel fenomeno di deviazione geodetica che studieremo tra poco.





