\chapter{Introduzione}

Queste dispense vogliono essere un'introduzione breve alla teoria della Relatività.
Alla fine di questa introduzione dovreste esservi resi conto che lo spazio ed il tempo, seppure nella vita di tutti i giorni sembrino cose
molto diverse, in realtà possono essere descritti in maniera unificata (e più elegante) come un'unica entità detta ``spaziotempo''
quadridimensionale. Per di più questo spazio non è ``piatto'' ma curvo, ovvero non rispetta la geometria di Euclide (cioè tutto quello che 
vi hanno insegnato finora come ``geometria'') in prossimità di grandi masse o su grandi distanze (fortunatamente la geometria euclidea rimane
un'ottima approssimazione per tutti i casi quotidiani).

La teoria della Relatività (speciale e generale) è una delle teorie accettate dalla fisica corrente per spiegare il mondo 
che ci circonda. Vale la pena chiarire quale peso dare alla parola ``teoria'', quanto ho detto nel paragrafo precedente potrebbe aver fatto 
immaginare a qualcuno che queste sono teorie nel senso di ``astruse invenzioni che non hanno conseguenz nella nostra vita di tutti i
giorni''. Ma non è così! La verità, una verità imparata con grande stupore nella prima metà del novecento (o durante il terzo anno
del corso di laurea in Fisica), è che la Natura non segue necessariamente quello che ci sembra ovvio o intuitivo. Per citare un grande 
fisico del novecento, Richard P. Feynman\footnote{R. P. Feynman - ``QED the strange theory of light and matter'' - op. cit. - pag. 10}:
\begin{quotation}
``Spero che possiate accettare la Natura così com'è, cioè assurda''
\end{quotation}
quindi, se tutto quanto seguirà vi sembrerà privo di senso o strano, non preoccupatevi, lo è. 

In compenso, dal 1905 ad oggi la relatività è stata verificata sperimentalmente numerose volte e viene ancora verificata  
quotidianamente nei grandi acceleratori di particelle. Inoltre essa ha anche alcune applicazioni tecnologiche che, può darsi, 
avete in tasca proprio in questo momento\footnote{ogni rivelatore GPS funziona grazie alla relatività, e questi sono 
inseriti nella maggior parte degli smartphone, quindi, se ne avete uno, potete sostenere di ``avere la relatività in tasca''.}.

L'avvento della relatività inoltre ha cambiato il modo con intendiamo alcuni concetti come lo spazio ed il tempo
facendoci capire che, per velocità molto grandi o campi gravitazionali molto intesi, alcune delle convinzioni consolidatesi nei 
secoli (e verificate nella vita di tutti i giorni) sembrano cadere.

Questa rivoluzione del nostro modo di pensare ha segnato storicamente il passaggio dalla fisica classica a quella moderna, 
la figura di Einstein è diventata mitica, grazie anche al suo grande carisma personale, e per questo i testi
che parlano di relatività sono tantissimi e di ogni livello.

Vi riporto qui di seguito qualche indicazione bibliografica, ovvero qualche testo che, nel caso siate interessati, 
potete consultare per saperne di più:
\begin{itemize}
 \item ``Relatività e senso comune'' - Hermann Bondi - ed. Zanichelli - ISBN 9788808008206 - 1973
 \item ``Relatività: esposizione divulgativa / Albert Einstein ; e scritti di Descartes, Newton, Lobacevskij, Riemann, Helmholtz, Maxwell, Poincare,''
  - A. Einstein - a cura di Bruno Cermignani - Universale Bollati Boringhieri - 
  - ristampa 2004 (prima ed. 1967)  - ISBN 88-339-0266-8
 \item ``Relatività e fisica delle particelle elementari'' - S. Bergia - a cura di Gianluca Introzzi - Carocci editore
  - 2009  - ISBN 978-88-430-4770-3 - questo libro è molto più difficile dei precedenti ma è interessante per la quantità di riferimenti 
 bibliografici che contiene e per il livello di aggiornamento.
\end{itemize}

Se volete leggere un'introduzione divulgativa e storica dell'altra grande teoria del novecento, la meccanica quantistica, potete leggere

\begin{itemize}
 \item ``Trent'anni che sconvolsero la fisica - La storia della teoria dei quanti'' - G. Gamow - Zanichelli - 1966 - il libro si riferisce 
 alla nascita della meccanica quantistica e in particolare agli anni 1900 - 1930), ogni capitolo è  scritto come ritratto di un 
 grande fisico della prima metà del novecento.
\end{itemize}

Altri libri più divulgativi sono:
\begin{itemize}
 \item ``Flatlandia'' - Edwin A. Abbott - ed. Adelphi; 14 edizione (9 giugno 1993) - ISBN-10: 8845909824 - un romanzo ambientato in un immaginario mondo bidimensionale i cui abitanti sono figure geometriche.
 \item  Personaggi e scoperte della fisica - E. Segrè - ed. Mondadori (26 novembre 1996), Oscar saggi - ISBN-10: 880442026X - una storia completa della fisica scritta Emilio Segrè, premio Nobel per la fisica.
\end{itemize}
