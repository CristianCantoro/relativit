\chapter{Conferme sperimentali della relatività speciale}
\minitoc

\section{Paradossi}
Riassumiamo brevemente alcuni degli effetti paradossali predetti dalla Rela-
tività Generale che abbiamo trovato nelle sezioni precedenti, insieme ad altri
che non abbiamo analizzato e cui accenniamo solo brevemente.

\begin{itemize}
 \item Deviazione geodetica: a causa della curvatura dello spaziotempo due
geodetiche localmente vicine possono divergere; questo si traduce ad
esempio nel cambio di orientazione nel tempo di un giroscopio orbitante.
 \item Precessione dei perieli: i semiassi delle orbite planetarie ruotano
(``precedono'') nel tempo più di quanto la teoria della gravità newtoniana 
avesse predetto. La conferma sperimentale più importante è stata la precessione 
del perielio di Mercurio, la cui marcata orbita a ``rosetta'' è 
in linea con i calcoli teorici della Relatività Generale.
 \item Buchi neri: come previsto dalla teoria di Schwarzschild, i buchi neri
sono corpi aventi raggio di Schwarzschild maggiore del proprio raggio,
in cui le informazioni entranti non potranno mai più essere recuperate\footnote{
Più recentemente sono nate molte altre variegate teorie al riguardo.}.
 \item Deviazione dei raggi luminosi: a causa della gravità, anche le onde
luminose e i segnali radar incurvano la loro traiettoria. In particolare,
se lo spazio è incurvato dal campo gravitazionale solare, un raggio di
luce che passa nelle vicinanze del sole non compierà un cammino rettilineo: 
dunque le stelle la cui luce ci giunge passando vicino al sole
ci appariranno in una posizione leggermente deviata dalla loro posizione 
effettiva. Di questo fatto di cui si è avuta conferma sperimentale
osservando pulsar passare dietro al sole durante le eclissi di sole.
 \item Dilatazione gravitazionale dei tempi: il tempo scorre più lentamente 
ove il campo è piè intenso.
Per convincerci di ciò, consideriamo la figura XXX: A è un osservatore 
posto al centro di un disco rotante, B è un osservatore
posto sul bordo del disco rotante, C è un osservatore inerziale posto
a terra, al di fuori del disco rotante. Dato che A e C sono in quiete
relativa, vedranno scorrere il tempo nello stesso modo. Dato inoltre che
B è in moto, nel giudizio di C, allora C vedrà l'orologio di B battere
più lentamente, e dunque anche A dovrà vedere l'orologio di B battere
più lentamente.
D'altro canto, anche A e C sono in quiete relativa, poiché sono entrambi
sulla piattaforma rotante. E se per C il fatto che l'orologio di B batta
più lentamente si pu` spiegare quindi con il fatto che B è in moto
relativo rispetto a C, questo non regge più per A, poiché non v'è alcun
moto relativo tra A e B sulla disco, c'é solo un campo di accelerazioni
centripete verso il centro del disco. Ne discende che, nel giudizio di
A, il tempo scorre più lentamente verso il bordo, e ricordando che
l'accelerazione centripeta è data da $2 \omega r$, possiamo anche generalizzare
dicendo che gli orologi battono più lentamente laddove il campo è più
forte.
 \item Lunghezza delle circonferenze: dall'esperimento mentale appena
esposto si deduce anche un altro fatto curioso. Se posizioniamo (a
riposo) dei metri lungo la circonferenza del disco e lungo il suo diametro,
troveremo che naturalmente il rapporto tra il numero di metri usati sarà
$\pi$. Se immaginiamo ora che la piattaforma si metta a ruotare, mentre
i metri posti lungo il diametro (ortogonali al moto) non cambieranno
lunghezza, i metri posti lungo la circonferenza (longitudinali al moto) si
accorceranno, e dunque avremo bisogno di più metri per coprire tutta
la circonferenza. Esibendo un rapporto tra circonferenza e diametro
che non è più il valore costante $\pi$, questo esperimento curioso mostra
anche che la geometria della Relatività Generale non può più essere la
geometria euclidea o semieuclidea.
 \item Red-shift gravitazionale: fenomeno direttamente derivante dalla dilatazione 
gravitazionale dei tempi. A causa della deformazione spazio-temporale 
prodotta dal campo gravitazionale solare, ad esempio, vediamo 
le righe degli spettri degli atomi solari eccitati a una frequenza
minore di quella che vedremmo per atomi eccitati sulla terra.
 \item Espansione dell'universo: le prime soluzioni cosmologiche alle equa-
zioni di campo di Einstein prevedevano un universo in espansione, fatto
confermato dagli esperimenti di Edwin Hubble nel 1921.
\end{itemize}

\section{Un'applicazione: il GPS}

Il Selective Availability è un sistema di generazione di errori voluto espressamente dal Dipartimento della Difesa 
statunitense per consentire il Global Positioning System (GPS) per usi civili.

Nel 1991, infatti, venne reso disponibile per usi civili il sistema Standard Positioning System (SPS), 
che introduceva volutamente degli errori mediante il Selective Availability in modo da rendere inaccurati i valori.

Dal 2000 si è rinunciato alla Selective Availability e l'SPS è divenuto accurato quasi quanto il 
Precision Positioning System (PPS), usato per scopi militari.

Gli orologi satellitari sono affetti dalle conseguenze della teoria della relatività. 
Infatti, a causa degli effetti combinati della velocità relativa, che rallenta il tempo sul satellite di circa 
7 microsecondi al giorno, e della minore curvatura dello spaziotempo a livello dell'orbita del satellite, 
che lo accelera di 45 microsecondi, il tempo sul satellite scorre ad un ritmo leggermente più veloce che a terra, 
causando un anticipo di circa 38 microsecondi al giorno, e rendendo necessaria una correzione automatica da parte 
dell'elettronica di bordo. 

Questa osservazione fornisce un'ulteriore prova dell'esattezza della teoria einsteniana 
in un'applicazione del mondo reale. L'effetto relativistico rilevato è, infatti, esattamente corrispondente a quello 
calcolabile teoricamente, almeno nei limiti di accuratezza forniti dagli strumenti di misura 
attualmente disponibili. Possono, inoltre, esistere altri tipi di errori del GPS che sono appunto 
di tipo atmosferico e di tipo elettronico.

È di fondamentale importanza ribadire che quello che permette al GPS la precisione cui 
si arriva (incertezze di pochi metri), sono proprio le correzioni di relatività generale; 
quelle della relatività ristretta, che ignorerebbero l'effetto sugli orologi dei satelliti, 
darebbero incertezze dell'ordine del chilometro che renderebbero il sistema del tutto inutile.

\url{http://www.astronomy.ohio-state.edu/~pogge/Ast162/Unit5/gps.html}
\url{http://it.wikipedia.org/wiki/Selective_Availability}

\section{Guardando avanti}

In un certo senso, questo è solo l'inizio della Relatività Generale, il cui ulteriore 
studio si intreccia poi naturalmente con la cosmologia, per cui la
Relatività Generale costituisce un ottimo modello geometrico gravitazionale,
un modello che ha superato indenne tutti i test sperimentali seri cui è stato
sottoposto negli ultimi decenni.

D'altro canto, il problema più serio è che la Relatività Generale è in aperto
conflitto con la meccanica quantistica; le predizioni relativistiche che sembrano 
valere ``nel grande'' sicuramente non valgono ``in piccolo'': la necessità
dei fisici è quindi ora quella di unificare questi due mondi così diversi 
(Relatività e Meccanica Quantistica), riuscendo a costruire una teoria coerente e
consistente, che possa inglobarle con successo.